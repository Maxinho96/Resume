%%%%%%%%%%%%%%%%%%%%%%%%%%%%%%%%%%%%%%%
% Deedy - One Page Two Column Resume
% LaTeX Template
% Version 1.2 (16/9/2014)
%
% Original author:
% Debarghya Das (http://debarghyadas.com)
%
% Original repository:
% https://github.com/deedydas/Deedy-Resume
%
% IMPORTANT: THIS TEMPLATE NEEDS TO BE COMPILED WITH XeLaTeX
%
% This template uses several fonts not included with Windows/Linux by
% default. If you get compilation errors saying a font is missing, find the line
% on which the font is used and either change it to a font included with your
% operating system or comment the line out to use the default font.
% 
%%%%%%%%%%%%%%%%%%%%%%%%%%%%%%%%%%%%%%
% 
% TODO:
% 1. Integrate biber/bibtex for article citation under publications.
% 2. Figure out a smoother way for the document to flow onto the next page.
% 3. Add styling information for a "Projects/Hacks" section.
% 4. Add location/address information
% 5. Merge OpenFont and MacFonts as a single sty with options.
% 
%%%%%%%%%%%%%%%%%%%%%%%%%%%%%%%%%%%%%%
%
% CHANGELOG:
% v1.1:
% 1. Fixed several compilation bugs with \renewcommand
% 2. Got Open-source fonts (Windows/Linux support)
% 3. Added Last Updated
% 4. Move Title styling into .sty
% 5. Commented .sty file.
%
%%%%%%%%%%%%%%%%%%%%%%%%%%%%%%%%%%%%%%%
%
% Known Issues:
% 1. Overflows onto second page if any column's contents are more than the
% vertical limit
% 2. Hacky space on the first bullet point on the second column.
%
%%%%%%%%%%%%%%%%%%%%%%%%%%%%%%%%%%%%%%


\documentclass[]{deedy-resume-openfont}
\usepackage{fancyhdr}
 
\pagestyle{fancy}
\fancyhf{}
 
\begin{document}

%%%%%%%%%%%%%%%%%%%%%%%%%%%%%%%%%%%%%%
%
%     LAST UPDATED DATE
%
%%%%%%%%%%%%%%%%%%%%%%%%%%%%%%%%%%%%%%
%\lastupdated

%%%%%%%%%%%%%%%%%%%%%%%%%%%%%%%%%%%%%%
%
%     TITLE NAME
%
%%%%%%%%%%%%%%%%%%%%%%%%%%%%%%%%%%%%%%
\namesection{Massimiliano}{Bruni}{
Email: \href{mailto:massimiliano.bruni@icloud.com}{massimiliano.bruni@icloud.com} | Cell: 334 924 0665
}

%%%%%%%%%%%%%%%%%%%%%%%%%%%%%%%%%%%%%%
%
%     COLUMN ONE
%
%%%%%%%%%%%%%%%%%%%%%%%%%%%%%%%%%%%%%%

\begin{minipage}[t]{0.33\textwidth} 

%%%%%%%%%%%%%%%%%%%%%%%%%%%%%%%%%%%%%%
%     Profilo
%%%%%%%%%%%%%%%%%%%%%%%%%%%%%%%%%%%%%%

\section{Profilo}
Sono un giovane laureando. Ho conseguito la Laurea Triennale in Ingegneria Informatica con 110 e lode ed ora sto per conseguire anche la Magistrale. Ho anche frequentato un master, la Advanced School in Artificial Intelligence al Consiglio Nazionale delle Ricerche. Ho un forte interesse verso l'informatica applicata al Machine Learning. Il mio obiettivo è diventare un Machine Learning Engineer, una figura con competenze sia informatiche che matematiche.
\sectionsep

%%%%%%%%%%%%%%%%%%%%%%%%%%%%%%%%%%%%%%
%     Link
%%%%%%%%%%%%%%%%%%%%%%%%%%%%%%%%%%%%%%

\section{Link} 
Github://\href{https://github.com/Maxinho96}{\bf Maxinho96} \\
LinkedIn://\href{https://www.linkedin.com/in/massimiliano-bruni-352926120}{\bf Massimiliano Bruni} \\
YouTube://\href{https://www.youtube.com/channel/UCqskrALDsaUvYC8ztJyqCug}{\bf Massimiliano Bruni} \\
Skype://\href{https://join.skype.com/invite/w9MIsgXZsji7}{\bf live:.cid.52cf21b3deb3210a}
\sectionsep

%%%%%%%%%%%%%%%%%%%%%%%%%%%%%%%%%%%%%%
%     Lingue
%%%%%%%%%%%%%%%%%%%%%%%%%%%%%%%%%%%%%%

\section{Lingue}
\textbf{Italiano}: Madrelingua \\
\textbf{Inglese}: Livello intermedio (B2)
\sectionsep

%%%%%%%%%%%%%%%%%%%%%%%%%%%%%%%%%%%%%%
%     Competenze
%%%%%%%%%%%%%%%%%%%%%%%%%%%%%%%%%%%%%%

\section{Competenze}

\subsection{Hard Skills}
\textbf{Avanzato:} Python \\
\textbf{Intermedio:} Java \textbullet{} SQL \textbullet{} Keras \\
TensorFlow \textbullet{} Scikit-learn \textbullet{} OpenCV \\
Matplotlib \textbullet{} NumPy \textbullet{} Git \textbullet{} Linux \textbullet{} macOS \\
\textbf{Base:} AWS \textbullet{} Hadoop \textbullet{} Spark
\sectionsep

\subsection{Soft Skills}
\textbf{Problem solving}, grazie agli studi in Ingegneria. \\
\textbf{Lavoro in team}, grazie ai vari progetti universitari e di stage. \\
\textbf{Capacità comunicative}, grazie ad una esperienza come tutor privato ed una come studente assistente bibliotecario.
\sectionsep

%%%%%%%%%%%%%%%%%%%%%%%%%%%%%%%%%%%%%%
%     Esami rilevanti
%%%%%%%%%%%%%%%%%%%%%%%%%%%%%%%%%%%%%%

\section{Esami rilevanti}
\textbf{Intelligenza Artificiale}: 30 e lode \\
\textbf{Machine Learning}: 30 e lode \\
\textbf{Sistemi Intelligenti per Internet}: 30 e lode \\
\textbf{Big Data}: 28 \\
\textbf{Analisi e Gestione dell'informazione su web}: 30
\sectionsep

%%%%%%%%%%%%%%%%%%%%%%%%%%%%%%%%%%%%%%
%     Hobby
%%%%%%%%%%%%%%%%%%%%%%%%%%%%%%%%%%%%%%

\section{Hobby}
Sport \textbullet{} Notizie di tecnologia \textbullet{} Videogiochi \\
Giochi da tavolo \textbullet{} Manga

%%%%%%%%%%%%%%%%%%%%%%%%%%%%%%%%%%%%%%
%
%     COLUMN TWO
%
%%%%%%%%%%%%%%%%%%%%%%%%%%%%%%%%%%%%%%

\end{minipage}
\hfill
\begin{minipage}[t]{0.66\textwidth} 

%%%%%%%%%%%%%%%%%%%%%%%%%%%%%%%%%%%%%%
%     Esperienze
%%%%%%%%%%%%%%%%%%%%%%%%%%%%%%%%%%%%%%

\section{Esperienze}
\subsection{Istituto di Scienze e Tecnologie della Cognizione - CNR}
\descript{Tirocinante in Machine Learning e Computer Vision}
\location{Marzo 2018 - Luglio 2018 | Roma}
Tirocinio svolto nell'ambito della Tesi Triennale e collegato al progetto descritto sotto chiamato \textit{Robot che riconosce oggetti}. \\
Tecnologie: \textbf{Python} \textbullet{} \textbf{TensorFlow} \textbullet{} \textbf{Scikit-learn} \textbullet{} \textbf{OpenCV} \textbullet{} \textbf{Matplotlib} \textbullet{} \textbf{Numpy}
\sectionsep

\subsection{Softlab}
\descript{Web and database developer (alternanza scuola-lavoro)}
\location{Febbraio 2015, una settimana | Roma}
Attività di stage in web development, riguardante lo sviluppo in team di un progetto. \\
Tecnologie: \textbf{HTML} \textbullet{} \textbf{CSS} \textbullet{} \textbf{Javascript} \textbullet{} \textbf{PHP} \textbullet{} \textbf{MySQL}
\sectionsep

%%%%%%%%%%%%%%%%%%%%%%%%%%%%%%%%%%%%%%
%     Istruzione
%%%%%%%%%%%%%%%%%%%%%%%%%%%%%%%%%%%%%%

\section{Istruzione} 

\subsection{Università degli Studi Roma Tre}
\descript{Laurea Magistrale in Ingegneria Informatica}
\location{Ottobre 2018 - Luglio 2020 | Roma}
\location{Voto atteso: 110 e lode}
\sectionsep

\subsection{Istituto di Scienze e Tecnologie della Cognizione - CNR}
\descript{Advanced School in Artificial Intelligence}
Master in Machine Learning Applications \\
\location{Ottobre 2019 - Aprile 2020 | Roma}
\sectionsep

\subsection{Università degli Studi Roma Tre}
\descript{Laurea Triennale in Ingegneria Informatica}
\location{Settembre 2015 - Luglio 2018 | Roma}
\location{Voto: 110 e lode}
\sectionsep

%%%%%%%%%%%%%%%%%%%%%%%%%%%%%%%%%%%%%%
%     Certificazioni
%%%%%%%%%%%%%%%%%%%%%%%%%%%%%%%%%%%%%%

\section{Certificazioni}
\begin{tabular}{@{}lll@{}}
\textbf{Anno} & \textbf{Ente} & \textbf{Certificato} \\
2017          & Coursera      & \textit{Machine Learning} \\
2015	      & Cambridge     & \textit{B2 First} \\
2015	      & Cisco         & \textit{CCNA Discovery: Networking for Home and Small Businesses} \\
2013	      & AICA          & \textit{ECDL} \\
\end{tabular}
\sectionsep

%%%%%%%%%%%%%%%%%%%%%%%%%%%%%%%%%%%%%%
%     Progetti
%%%%%%%%%%%%%%%%%%%%%%%%%%%%%%%%%%%%%%

\section{Progetti (Titoli cliccabili)}

\descript{Sistema di videosorveglianza (\href{https://github.com/Maxinho96/person\_identification}{Link})}
Per la Tesi Magistrale ho realizzato un sistema di videosorveglianza che rileva le persone e determina anche le loro identità dato un dataset di persone note.

\descript{Object Detection con drone (\href{https://youtu.be/cnnGjHns818}{Link})}
Per la Advanced School in Artificial Intelligence ho realizzato un sistema di object detection sulla camera di un drone.

\descript{Bot Telegram: classificatore di immagini (\href{https://github.com/Maxinho96/Image-Classifier-Telegram-Bot}{Link})}
Come progetto personale ho realizzato un bot Telegram a cui si può inviare un'immagine per sapere quale oggetto vi è rappresentato.

\descript{Robot che riconosce oggetti (\href{https://github.com/Maxinho96/Machine-Learning-Research-Project}{Link})}
Per la Tesi Triennale ho realizzato un sistema pensato per funzionare su un robot che riconosce oggetti comuni ed impara autonomamente nuovi oggetti.

\end{minipage} 
\end{document}  \documentclass[]{article}
